\documentclass[11pt]{article}
\hoffset -0.5in
\voffset -1in
\textheight 9in
\textwidth 6in
\pagestyle{plain}

\usepackage{graphicx}
\usepackage[mathscr]{eucal}
\usepackage{amsmath}
\usepackage{amssymb}
\usepackage{amsthm}
\usepackage{amscd}
\usepackage{verbatim}

%\usepackage{setspace}
%\onehalfspacing
%\doublespacing

\usepackage{color}
\DeclareMathOperator*{\argmax}{arg\,max}

\title{Research Documentation}
\author{Onkur Sen, Kurt Stallmann}
\date{}

\begin{document}
\maketitle

\section{Determining the Key of an Input Score}
Our task in this module is to determine the tonic and mode (major/minor) for a score {\it given only the melodic line and not the harmonic progression}. This poses a difficult problem as often times ambiguity arises when considering only the melody; the presence of the harmony guides the melody towards a certain key. However, we can use the ambiguity to our advantage when considering generating variation upon an original work. Our approach is broken up into steps:

\subsection{Note-Counting and the Characteristic Triad}
We assume the musical work is sampled at some regular time interval, i.e., we are assuming a discrete-time input signal $M$. We simply count up the number of occurrences of each pitch in the chromatic scale. This indicates the dominant contributions among the scale to the character of the piece and signals a first step towards understanding the key of the piece.

Mathematically, we represent pitches on a chromatic scale as 12 values between 0 and 11, with 0 being C and 11 being B, in accordance with the {\bf MIDI standard}. For our purposes, notes that differ by octaves are considered equivalent and thus undergo a modulo 12 operation. We can thus define the {\bf note-counting operation} on a pitch $p$ as:
\[n(p) = \#\{x\in S; x = p\}, \quad p=0,1,\ldots,11. \]

We form a triad from the three most frequently-occurring pitches as determined by the note-counting step before. More explicitly, let us assume that the 12 pitches form a set $P=\{p_i; i = 1,2,\ldots,12\}$ such that $n(p_1)>n(p_2)>\ldots>n(p_{12})$. Then the {\bf characteristic triad} of the piece is the set $T=\{p_1, p_2, p_3\}$.

To distinguish the individual pitches within the triad, however, we introduce a {\bf weighting function} that gives higher priority to the more-frequent notes in the piece. This is defined as follows:
\[w(p_i) = \frac{n(p_i)}{\sum_{j=1}^3 n(p_j)}, \quad i = 1, 2, 3.\]

\subsection{Triad-Matching}

We have a preexisting {\bf chord set} $C$ containing major and minor chords in all keys within the chromatic scale. Furthermore, each chord $c\in C$ has a {\bf tonic} $t(c)$ as well as a {\bf mode} $m(c)$, which can be either major or minor. This serves as our comparison mechanism between a triad and the key of a piece. If $T\in C$, then $t(M)=t(T)$ and $m(M)=m(T)$.

\subsection{Dyad-Matching}

If triad-matching is unsuccessful, then we reduce our considerations to pinpoint the more fundamental contributions of the piece to the key. Thus, we define the {\bf characteristic dyad} $D=\{p_1,p_2\}$ as the two most frequently-occurring notes in the piece. Then if $D\in c'$ for a chord $c'\in C$, then it is easy to show that we can uniquely establish the tonic as $t(M)=t(c')$. 

However, the mode $m(M)$ is not necessarily unique; to find it, we compare the frequency of the major and minor thirds with respect to the tonic, which are $n(t(D)+4)$ and $n(t(D)+3)$, respectively, and assign the mode based on which third occurs more. However, if the two counts are equal, then the $m(D)$ cannot be determined and is left as a question mark, ``?".

{\color{red} 
\noindent Compare sixths as tiebreaker to increase sophistication?\\
Add sum of differences of thirds and sixths and see if $>, <, =$ 0?\\
}

\subsection{Scale-Matching}
If both triad-matching and dyad-matching are unsuccessful, then to reduce our scope further and consider a single note essentializes the piece into a single contribution and is not a sophisticated approach. Rather, we now compare the characteristic dyad to a {\bf scale set} $S$ that contains major and minor scales for all pitches in the chromatic scale. Major scales are defined the usual way; for instance, the C major scale is $s_{C,major} = \{0,2,4,5,7,9,11\}$, corresponding to \{C,D,E,F,G,A,B\}. Minor scales, however, are defined so that they include both the natural and harmonic minor possibilities. Thus, the C minor scale is $s_{C,minor} = \{0,2,3,5,7,8,10,11\}$, corresponding to \{C,D,E$^{\flat}$,F,G,A$^{\flat}$,B$^{\flat}$,B\}.

We construct a {\bf possible key set} as only those keys for which the characteristic dyad is part of the scale:
\[K= \{(t(s),m(s)); s\in S \wedge D\subset s\}.\]

We have figured out two ways to determine the key from here as detailed below:

\subsubsection{Maximizing the Correlation Function}
We first define a {\bf consonance relation} between two pitches in accordance with traditional music theory that assesses the euphony of the interval determined between the two pitches:
\begin{equation*}
\Gamma(p, p') = \left\{
\begin{array}{ccc}
1 & \text{\emph{if}} & p = p' \text{ OR } p, p' \text{ form a perfect } 5^{th} \\
\frac{1}{2} & \text{\emph{if}} & p, p' \text{ form a perfect } 4^{th} \text{ OR major/minor } 3^{rd} \text{ OR major/minor } 6^{th}\\
0 & \text{\emph{if}} & p, p' \text{ form a } 2^{nd} \text{ OR } 7^{th}\\
\end{array}
\right.
\end{equation*}

Note that the intervals here are assumed to be in the same octave; to account for different octaves, we perform a modulo 12 operation on both pitches. Furthermore, it is important to note that the interval is defined as {\em the distance from p to p'}. This resolves any confusion between whether the orientation of an interval produces, e.g., a perfect 5$^{th}$ or a perfect 4$^{th}$. More explicitly, the interval C-G is a perfect 5$^{th}$, whereas the interval G-C is a perfect 4$^{th}$.

We define the {\bf correlation function} of the characteristic triad $T$ and any scale $s$ as:
\[F_T(s) = \sum_{i=1}^3 w(p_i) \cdot \Gamma(t(s), p_i),\]
where, as usual, $t(s)$ represents the tonic of the scale. Note again that the intervals here are defined {\em with respect to the tonic of the scale}.

We then seek to maximize $F_T(s(k))$, where $s(k)$ represents the scale generated by each key $k$ in the possible key set $K$. More explicitly, within the scales corresponding to the possible keys obtained before, the value of the correlation function weights more frequent notes in the triad that appear in the scale. Thus, if the more frequently-occurring notes form a consonant relationship with the tonic of the key, then the key is likely to be a more appropriate choice for the piece. This method is guaranteed to determine a key for a given input score $M$.

\subsubsection{Maximizing the Characteristic Number}

We define the {\bf characteristic number} of a key $k\in K$ as 
\[\sigma(k) = \#(c(k) \cap K),\] 
where $c(k)$ denotes the chord generated by the tonic and mode contained in the key $k$. Thus, the characteristic number of a key represents how many of the keys in the possible key set are represented in the root chord of that key. The pivotal realization in this approach is that {\it we seek to maximize the characteristic number in order to obtain the final key of a given piece.} This is because by doing so, we choose a key that maximally reflects the given possibilities and thus gives us a more correct answer. Thus, we can define the key of the input piece simply as follows:
\[k(M) = (t(M), m(M)) = \underset{k\in K}{\argmax}\; \sigma(k).\]
However, note that this approach may result in multiple key choices with an equal characteristic procedure. In that case, the relation above is applied {\bf recursively} on a new possible key set which contains only those keys that maximize $\sigma(k)$. It can be shown easily that this procedure need only be applied once more to obtain a definite result in the same way in a similar manner that the characteristic dyad uniquely determines a key if it matches a chord $c\in C$. This method is guaranteed to determine a key for a given input score $M$.
\end{document}